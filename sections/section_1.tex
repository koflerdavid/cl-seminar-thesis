\section{Introduction}

Finite automata are a basic tool of computer science, and among their
applications are lexical analyzers, software engineering, model checking,
string matching algorithms and others.
Though many of these areas use more powerful constructions
(for example, regular expression engines usually also implement backreferences
and other extensions),
the properties of their foundations and their analysis is still important.

An important question about two finite automata is whether they are equivalent,
that is, whether they accept the same language.
It is not obvious how to do this efficiently since it is a PSPACE-problem for nondeterminstic automata.
In their work, Bonchi and Pous developed a coinductive framework for analyzing
and solving this problem, and showed that older algorithms and some related ones can
be presented using this framework.
Also, with this background they developed improvements which significantly improve on previous
algorithms~\cite{abdulla2010simulation,doyen2010antichain}, according to
their experimental evaluation.

This thesis first reviews some definitions of the underlying theories, then
it will present the contributions of Bonchi and Pous.
Then the results of their experiments will be presented and interpreted,
and finally future research directions are outlined.
