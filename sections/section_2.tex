\section{Equivalence of nondeterministic finite automata}

\subsection{Finite automata}

A Nondeterministic Finite Automaton $N = \langle Q_N, \Sigma_N, F_N, \delta_N \rangle$ consists of five parts:

\begin{itemize}
  \item a \textit{finite} set of states $Q_N$,
  \item a \textit{finite} set of symbols $\Sigma_N$, called the input alphabet,
  \item a set of accepting states $F_N$, which has to be a subset of $Q_N$, and
  \item a transition function $\delta_N : (Q_N, \Sigma_N) \to \mathcal{P}(Q_N)$ mapping a state and an input symbol
          to a set of successor states.
\end{itemize}

The purpose of an automaton is to accept or to reject a word
composed of symbols from the input alphabet.
A finite automaton does this by starting out from one of the
states and then using every symbol of the input word (in the given
order) and the mapping $\delta_N$ to reach a successor state.
In general there can be more than one (or no) ways to proceed since $\delta_N$ returns a set of possible successor states.
Formally, this is what the function $\hat{\delta}_N$ does:

\begin{definition}
  \begin{align*}
    \hat{\delta}_N(p, \epsilon) &= \{p\}\\
    \hat{\delta}_N(p, aw) &= \bigcup_{q \in \delta_N(p, a)}\hat{\delta}_N(q, w)
  \end{align*}
\end{definition}

A finite automata accepts an input word if it is possible to reach one of the
accepting states by using all of the word's characters. The set of words which
an automaton accepts is called its language:

\begin{definition}
  $\mathcal{L}(N) = \{ w \forWhich \exists q \in Q_N: \hat{\delta}_N(q, w) \cap F_N \neq \emptyset \}$
\end{definition}

For the purposes of this paper it is interesting to start from an arbitrary set of states of the automaton.

\begin{definition}
  $\mathcal{L}_N(P) = \{ w \forWhich \exists p \in P: \hat{\delta}_N(p, w) \cap F_N \neq \emptyset\}$
\end{definition}

A Deterministic Finite Automaton $D$ is special in that $\delta_M$ assigns to
every state and input symbol exactly one successor state. Every
nondeterministic automata $N$ can be converted into the definite automaton
$N^\# = \langle \mathcal{P}(Q_N), \Sigma_N, F_N^\#, \delta_N^\# \rangle$:

\begin{definition}
  \begin{align*}
    F_N^\# &= \{ S \in \mathcal{P}(Q) \forWhich S \cap F \neq \emptyset \}\\
    \delta_N^\#(P, a) &= \bigcup_{p \in P} \delta_N(p, a)
  \end{align*}
\end{definition}
