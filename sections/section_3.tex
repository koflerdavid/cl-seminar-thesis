\section{Finite automata}

This section serves to review the theory of finite automata, to define the
problem of checking equivalence, and to highlight its challenges.

\subsection{Nondeterministic finite automata}

\begin{definition}
  A \emph{Nondeterministic Finite Automaton} (NFA) $N = \langle Q_N, \Sigma_N, S_N, F_N, \delta_N \rangle$ consists of five parts:

  \begin{itemize}
    \item a finite set of states $Q_N$,
    \item a finite set of symbols $\Sigma_N$, called the input alphabet,
    \item a set of starting states $S_N \subseteq Q_N$,
    \item a set of accepting states $F_N \subseteq Q_N$, and
    \item a transition function $\delta_N : (Q_N \times \Sigma_N) \to \mathcal{P}(Q_N)$ mapping a state and an input symbol
            to a set of successor states.
  \end{itemize}
\end{definition}

The purpose of an automaton is to accept or to reject a word
composed of symbols from the input alphabet.
A finite automaton does this by starting out from one of the
states and then using every symbol of the input word (in the given
order) and the mapping $\delta_N$ to reach a successor state.
In general there can be more than one (or no) ways to proceed since $\delta_N$ returns a set of possible successor states.
Formally, this is what the function $\hat{\delta}_N$ does:

\begin{definition}
  \begin{align*}
    \hat{\delta}_N(p, \epsilon) &:= \{p\}\\
    \hat{\delta}_N(p, aw) &:= \bigcup_{q \in \delta_N(p, a)}\hat{\delta}_N(q, w)
  \end{align*}
\end{definition}

\begin{definition}
  The language accepted by the set of states $P$ is the set of words which can
  make the automaton reach a final state by starting from one of the members
  of $P$.\\
  The language accepted by an automaton is the language accepted by the set of
  its starting states.

  \begin{align*}
    \mathcal{L}_N(P) &:= \{ w \forWhich \exists p \in P: \hat{\delta}_N(p, w) \cap F_N \neq \emptyset\}\\
    \mathcal{L}(N) &:= \mathcal{L}_N(S_N)
  \end{align*}
\end{definition}

% A finite automata accepts an input word if it is possible to reach one of the
% accepting states by using all of the word's characters.
% The set of words which an automaton accepts if starting from one of the states
% in the set $P$ is called the language of $P$:
%
% \begin{definition}
%   $\mathcal{L}(N) = \{ w \forWhich \exists s \in S_N: \hat{\delta}_N(s, w) \cap F_N \neq \emptyset \}$
% \end{definition}
%
% For the purpose of this thesis it is interesting to start from an arbitrary set of states of the automaton.


\subsection{Deterministic Finite Automata}

\begin{definition}
  A \emph{Deterministic Finite Automaton} (DFA)
  $D = \langle Q_D, \Sigma_D, S_D, F_D, \delta_N \rangle$
  consists of five parts:

  \begin{itemize}
    \item a finite set of states $Q_D$,
    \item a finite set of symbols $\Sigma_D$, called the input alphabet,
    \item a starting state $S_D \in Q_D$,
    \item a set of accepting states $F_D \subseteq Q_D$, and
    \item a total transition function
            $\delta_D : (Q_D \times \Sigma_D) \to Q_D$,
             mapping a state and an input symbol to a successor state.
  \end{itemize}

\end{definition}

The differences to a NFA are that there is only one starting state, and that
the transition function produces only a single state.
Therefore, there is always only one possibility to transition to a successor
state.

\begin{definition}
  Every NFA $N$ can be converted into the DFA
  $N^\# = \langle \mathcal{P}(Q_N), \Sigma_N, \{ S_N \}, F_{N^\#}, \delta_{N^\#} \rangle$:

  \begin{align*}
    F_{N^\#} &:= \{ P \in \mathcal{P}(Q) \forWhich P \cap F \neq \emptyset \}\\
    \delta_{N^\#}(P, a) &:= \bigcup_{p \in P} \delta_N(p, a)
  \end{align*}
\end{definition}

It is important to notice that the cardinality of the set of states of a
determinised NFA is $\mathcal{O}(2^{|Q|})$.
This can quickly become a big problem if algorithms and representations of
automata are not chosen properly, as the following subsection will show.

\subsection{Minimization and equivalence of finite automata}

Within the above framework it is easy to define automata which have inaccessible
states, that is, there is no input word which can reach one of those states.
Also, it is possible to introduce redundancies into an automata.

\begin{example}
  These two automata accept exactly the same language:\\
  \center
  \begin{tikzpicture}[shorten >=1pt,node distance=2.5cm,on grid,auto]
    \node[state,initial,accepting] (p) [] {$p$};
    \node[state,initial,accepting] (q_0) [right=of p] {$q_0$};
    \node[state,accepting] (q_1) [right=of q_0] {$q_1$};
    \path[->]
      (p) edge [loop above] node {a} ()
      (q_0) edge [bend left] node {a} (q_1)
      (q_1) edge [bend left] node {a} (q_0);
  \end{tikzpicture}
\end{example}

Since many useful applications of automata involve solving PSPACE-complete
decision problems, it is desirable to minimize the number of states of an
automaton.
For DFAs this is a well-researched topic, and there are efficient algorithms.
It is not so easy for NFAs however: the simplest approach would be to
minimize it after turning the automaton into a deterministic one, which in the
worst case causes exponential (in the number of states of the NFA)
runtime complexity.

We get similar results for determining equivalence of NFAs and DFAs:
Hopcroft and Karp developed an (almost) linear algorithm for the deterministic
case which runs in $\mathcal{O}(n \times \alpha(n))$ thanks to a
Union-Find data structure.\footnote{$\alpha$ is the inverse of Ackermann's
function, which grows extremely slowly.}
However, for the NFA case we are stuck with exponential runtime, for the same
reason as in the minimization case.

In both cases research focuses on looking for algorithms whose
(runtime or memory) complexity is exponential only in the worst case.
