\section{Bonchi and Pous' coinductive approach}

The main contribution of Bonchi and Pous is their presentation of existing
algorithms using coinductive frameworks. This makes it possible to use a more
precise performance metric, that is, the number of pairs they have to check.

Bonchi and Pous use a proof method originating in concurrency theory called
``bisimulation''. Most of the following theory originates from earlier work of
Milner~\cite{milner1989communication} and Sangiorgi~\cite{sangiorgi1998bisimulation}.

\subsection{Coinduction and Bisimulations}

Coinductive frameworks originate from concurrency theory and allow reasoning
about structures of which only their behaviour is known. The best-known
coinductive proof principle is bisimulation.

Bisimulations are relations on states of an automaton (or a similar transition
system, such as actors). They are special in that they are
congruent with the transition function of the automaton, i.e., if two states are
related, then their successors also have to be related.
For NFAs it is defined as follows:

\begin{definition}
  A bisimulation $R_N$ is a binary relation on the set of states of
  the automaton $N$ satisfying the following properties:
  \begin{align*}
    (P_1, P_2) \in R_N \iff &\text{acc}_N(P_1) \iff \text{acc}_N(P_2)\\
      &\land \forall a \in \Sigma_N: \delta_N(P_1, a) \setAsRelation{R_N} \delta_N(P_2, a)
  \end{align*}

  The notation $P_1 \sim P_2$ is used to express that there exists a bisimulation
  which relates {$P_1$ and $P_2$}.
\end{definition}



It is easy to see that any bisimulation progresses to itself.

\begin{definition}
  Two sets of states of an automaton accept the same language
  if there is a bisimulation relating them.

  \begin{align*}
    \mathcal{L}_N(P_1) = \mathcal{L}_N(P_2) \iff P_1 \sim P_2
  \end{align*}
\end{definition}

\subsection{Computing bisimulations}

Bisimulations on states of a deterministic automaton can be computed using the following algorithm.
To use it to compare two distinct automata, the automata have to be merged.
It is necessary to ensure that their alphabets are the same (easily satisfied by
using the union of the two alphabets), that their state sets are disjunct
(by renaming), and that each of the input automata has only one starting state.

\begin{definition}
  This is the naive algorithm for deciding equivalence of two states of an automaton.
  \begin{align*}
    \underline{\mathit{Naive(p, q)}}: &\quad Q \to Q \to \{\mathbf{true}, \mathbf{false}\} \\
    \text{(1) } & R := \emptyset, \mathit{todo} := \{(p, q)\} \\
    \text{(2) } & \text{while } \mathit{todo} \neq \emptyset \text{ do}\\
      & \text{(2.1)}\quad \mathit{todo} := \mathit{todo} \setminus \{(p', q')\}\\
      & \textbf{(2.2)}\quad \textbf{if } \mathbf{(p', q') \in} \textbf{ R} \textbf{ then goto (2)}\\
      & \text{(2.3)}\quad \text{if } p' \in F \nLeftrightarrow q' \in F \text{ then return } \mathbf{false}\\
      & \text{(2.4)}\quad \mathit{todo} := \mathit{todo} \cup \bigcup_{a \in \Sigma}{(\delta(p', a), \delta(q', a))}\\
      & \text{(2.5)}\quad R := R \cup \{(p', q')\} \\
    \text{(3) } & \text{return } \mathbf{true}\\
  \end{align*}
\end{definition}


The algorithm first computes a bisimulation. If it succeeds then the languages are
equal. If it fails to do so, the languages are not equal.
Finding a bisimulation, and therefore proving the language equivalence, yields
the worst-case runtime of the algorithm.

The algorithm can be extended to also find a witness for the inequality of the
input states. This can be done by tagging each pair in \textit{todo} with the
symbol sequence which yielded it. This results in the following algorithm,
which yields either \textbf{true} or a word $w$ for which
$\neg (\hat{\delta}(p, w) \in F \iff \hat{\delta}(q, w) \in F)$.

\begin{definition}
  \begin{align*}
    \underline{\mathit{NaiveWithWitness(p, q)}}: &\quad Q \to Q \to \{\mathbf{true}, \mathbf{\Sigma^\ast}\} \\
    \text{(1) } & R := \emptyset, \mathit{todo} := \{(p, q)_\epsilon\} \\
    \text{(2) } & \text{while } \mathit{todo} \neq \emptyset \text{ do}\\
      & \text{(2.1)}\quad \mathit{todo} := \mathit{todo} \setminus \{(p', q')_w\}\\
      & \text{(2.2)}\quad \text{if } (p', q') \in \text{ R} \text{ then goto (2)}\\
      & \text{(2.3)}\quad \text{if } p' \in F \nLeftrightarrow q' \in F \text{ then return } w\\
      & \text{(2.4)}\quad \mathit{todo} := \mathit{todo} \cup \bigcup_{a \in \Sigma}{(\delta(p', a), \delta(q', a))_{wa}}\\
      & \text{(2.5)}\quad R := R \cup \{(p', q')\} \\
    \text{(3) } & \text{return } \mathbf{true}\\
  \end{align*}
\end{definition}


\subsection{Bisimulations up-to}

Since in the worst case there are $\mathcal{O}(n^2)$ pairs of states of the
input automaton, it is advantageous to think about how to reduce the number of
pairs we have to compute.

\begin{definition}
A relation R is a bisimulation up-to $f$
($f : \mathcal{P}(Q \times Q) \to \mathcal{P}(Q \times Q$)) iff $R \rightarrowtail f(R)$.
\end{definition}

Sangiorgi introduced the notion of \textit{respectful} functions to describe
function which yield bisimulations up-to.
In an earlier work of Pous he introduced compatibility as a generalisation of
Sangiorgi's respectfulness~\cite{pous2007complete}, but in the context of Bonchi and Pous's paper they
are equivalent. Nevertheless, Bonchi and Pous are using this name in their work.

\begin{definition}
  A function $f : \mathcal{P}(Q \times Q) \to \mathcal{P}(Q \times Q)$ is compatible if it is
  \begin{itemize}
    \item monotone ($\forall R, R' \subseteq Q \times Q: R \subseteq R' \Rightarrow f(R) \subseteq f(R')$), and
    \item preserves progression:
        $\forall R, R' \subseteq Q \times Q: R \rightarrowtail R' \Rightarrow f(R) \rightarrowtail f(R')$.
  \end{itemize}
\end{definition}

A bisimulation $R$ up-to $f$ is a relation where $f(R)$ is a bisimulation.
The function $f$ has to be \textit{compatible}. Bonchi and Pous proved that
any bisimulation up-to is contained in a bisimulation. This is what their
work is about: making it possible to more easily compute bisimulations.

A main contribution of Bonchi and Pous is also to show that many nontrivial functions
are compatible. The most important of them are

\begin{itemize}
  \item the identity function
  \item pointwise union of a family of compatible functions: \\
          $\bigcup F(R) = \bigcup_{f \in F} f(R)$
  \item $f \circ g$ where $f$ and $g$ are compatible function
  \item the identity closure of $R \subseteq \mathcal{P}(Q \times Q): R \cup \{(q, q)\forWhich q \in Q\}$
  \item the symmetric closure of a relation: $s(R) = R \cup \{(q, p) \forWhich p \setAsRelation{R} q\}$
  \item squaring of a relation $R$: $P \setAsRelation{R} P' \land P' \setAsRelation{R} P'' \Rightarrow P \setAsRelation{R} P''$
  \item infinite iteration of a compatible function: $f^\omega(R) = \bigcup_{n = 1}^{\infty} f^n(R)$
  \item by the above four points, the equivalence closure
\end{itemize}

\subsection{Examples}

Hopcroft and Karp's algorithm for determining the equivalence of DFAs is
equivalent to computing a bisimulation up to equivalence, that is, exploiting
symmetry and transitivity.

Bonchi and Pous exploit the following property of the states of determinised NFAs:

\begin{corollary}
  Computing the language of a determinised NFA is a homomorphism:\\
  $\mathcal{L}(P_1) = \mathcal{L}(Q_1) \land \mathcal{L}(P_2) = \mathcal{L}(Q_2)
    \Rightarrow \mathcal{L}(P_1 \cup P_2) = \mathcal{L}(Q_1 \cup Q_2)$
\end{corollary}

After proving that this function is compatible a
\textit{bisimulation up to congruence} can be computed for determinised NFAs.
