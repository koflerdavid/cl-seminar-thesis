\section{Bonchi and Pous' coinductive approach}

The main contribution of Bonchi and Pous is their presentation of existing
  algorithms using coinductive frameworks.
This makes it possible to use a more precise performance metric, that is,
  the number of pairs they have to check.

Bonchi and Pous use a proof method originating in concurrency theory called
  ``bisimulation''. Most of the following theory originates from earlier work of
  Milner~\cite{milner1989communication} and
  Sangiorgi~\cite{sangiorgi1998bisimulation}.

In the following the presence of an NFA $N$ is assumed, for which the language
  equality of some states has to be proven.

\subsection{Coinduction and Bisimulations}

Coinductive frameworks originate from concurrency theory and allow reasoning
  about structures of which only their behaviour is known.
  The best-known coinductive proof principle is bisimulation.

Bisimulations are relations on states of an automaton (or a similar transition
  system, such as actors).
They are special in that they are congruent with the transition function of the
  automaton, i.e., if two states are related,
  then their successors also have to be related.
For NFAs it is defined as follows:

\begin{definition}
  A bisimulation $R_N$ is a binary relation on the set of states of
    the automaton $N$ satisfying the following properties:
  \begin{align*}
    (P_1, P_2) \in R_N \iff &(\text{acc}_N(P_1) \iff \text{acc}_N(P_2))\\
      &\land \forall a \in \Sigma_N: \delta_N(P_1, a) \setAsRelation{R_N} \delta_N(P_2, a)
  \end{align*}

  The notation $P_1 \sim P_2$ is used to express that there exists a bisimulation
    which relates {$P_1$ and $P_2$}.
\end{definition}

\subsection{Progression}

For the following definitions and proof we need the notion of progression which
  relates two relations on sets of states.

\begin{definition}
  A relation $R$ progresses to $R'$
    ($R, R' \subseteq \mathit{P}(Q_N) \times \mathit{P}(Q_N)$)
    if $P_1 \setAsRelation{R} P_2 \to$
  \begin{align*}
    \accepting{P} \iff \accepting{P}\\
    \land \forall a \in \Sigma_N: \delta_N(P, a) \setAsRelation{R'} \delta_N(P', a)
  \end{align*}
\end{definition}

It directly follows from its definition that any bisimulation progresses to itself.

\begin{theorem}
  Two sets of states of an automaton accept the same language
    if there is a bisimulation relating them.

  \begin{align*}
    \mathcal{L}_N(P_1) = \mathcal{L}_N(P_2) \iff P_1 \sim P_2
  \end{align*}
\end{theorem}

To prove this statement we need two auxiliary lemmas.

The first lemma is useful to prove equality of two sets defined in the
  intensional way, that is, by specifying that elements are members iff they
  fulfil a certain property.
It is useful for converting to and getting rid of set notation.

\begin{lemma}
  \begin{align*}
    \{ a \in M \forWhich p_1(a) \} = \{ a \in M \forWhich p_2(a)\} \iff (\forall a \in M: p_1(a) \iff p_2(a))
  \end{align*}
\end{lemma}

\begin{proof}
  We make use of the property that any two sets $A$ and $B$ are equal if
    $A \subseteq B$ and $B \subseteq A$.
  Certainly for all pair of intensionally defined sets
    $S_1 = \{a \in M \forWhich p_1(a)\}$ and
    $S_2 = \{a \in M \forWhich p_2(a)\}$ it is true that
    $S_1 \subseteq S_2 \iff p_1 \Rightarrow p_2$.
  If we combine these two statements then we get
  \begin{align*}
         &S_1 = S_2\\
    \iff &S_1 \subseteq S_2 \land S_2 \subseteq S_1\\
    \iff &p_1 \Rightarrow p_2 \land p_2 \Rightarrow p_1\\
    \iff &(p_1 \iff p_2)
  \end{align*}
\end{proof}

The second lemma makes it easier to reason about language concatenation.

\begin{lemma}
  \begin{align*}
    \forall R_1, R_2 &\subseteq \mathcal{P}(Q_N), \forall v \in \Sigma_N^*:\\
      &\lang(R_1) = \lang(R_2) \Rightarrow
        \lang(\hat{\delta}_N(R_1, v)) = \lang(\hat{\delta}_N(R_2, v))
  \end{align*}
\end{lemma}

\begin{proof}
  We use induction on the length of $v$ to prove the above statement.

  $v = \epsilon$:
    $\lang(\hat{\delta}_N(R_1, \epsilon))
      = \lang(R_1)
      = \lang(R_2)
      = \lang(\hat{\delta}_N(R_2, \epsilon))$

  $v = w \cdot a$:
    It follows from basic Kleene algebra axioms that
      for all languages $L_1$ and $L_2$ and symbols $a$,
      $L_1 = L_2 \Rightarrow L_1 \cdot a = L_2 \cdot a$.
    Using the induction hypothesis and the definition of the language of NFAs
      we get the lemma
    \begin{align*}\lang(P_1) &= \lang(P_2) \Rightarrow\\
      &\forall a \in \Sigma_N, w \in \Sigma_N^\ast:
        \accepting{\delta_N(\hat{\delta}_N(R_1, w), a)}
          \iff \accepting{\delta_N(\hat{\delta}_N(R_2, w), a)}.
    \end{align*}
    Using the alternative definition of $\hat{\delta}_N$ we get
      $\lang(\hat{\delta}_N(R_1, wa)) = \lang(\hat{\delta}_N(R_2, wa)$.
\end{proof}

Now we are ready for the main proof.

\begin{proof}
  To prove that two sets of states of the NFA $N$ have the same language
    iff they are related by a bisimulation we have to prove two directions:
  \begin{enumerate}
    \item[``$\Rightarrow$'']:
      It has to be proven that
        $\lang(P_1) = \lang(P_2) \Rightarrow P_1 \sim P_2$.
        Basically we have to prove that all pair of language-equal sets of
          states are related by a bisimulation.
      \begin{enumerate}
        \item $\accepting{P_1} \iff \accepting{P_2}$:
          This statement will be proven by contradiction.
          There are two cases two analyse, the first being
            $\accepting{P_1} \land \lnot \accepting{P_2}$.
          Since
            $\hat{\delta}_N(P_1, \epsilon) = P_1$ (therefore $\epsilon \in \lang(P_1)$) and
            $\hat{\delta}_N(P_2, \epsilon) = P_2$ (therefore $\epsilon \notin \lang(P_2)$),
            the languages $\lang(P_1)$ and $\lang(P_2)$ differ, which contradicts with
            our initial assumption that they are equal.
          Since the other case is symmetrical, we obtain a contradiction,
            which concludes this point.
        \item $\forall a \in \Sigma_N: \delta_N(P_1, a) \setAsRelation{R} \delta_N(P_2, a)$:
          Applying the definition of $R$ we get
            $\forall a \in \Sigma_N: \lang(\delta(P_1, a)) = \lang(\delta(P_2, a))$.
          This is equivalent to
            $\forall a \in \Sigma_N: \lang(\hat{\delta}_N(P_1, a)) = \lang(\hat{\delta}_N(P_2, a))$,
            which is an instance of the lemma proven above.
      \end{enumerate}
    \item[``$\Leftarrow$'']:
      Now it has to be proven that all pairs of sets of states related by a
        bisimulation accept the same language, that is
        $P_1 \sim P_2 \Rightarrow \lang(P_1) = \lang(P_2)$,
        which is equivalent to
        $\forall w \in \Sigma_N^\ast:
          \accepting{\hat{\delta}_N(P_1, w)} \iff \accepting{\hat{\delta}_N(P_2, w)}$.
      To do so, induction on the length of $w$ is used.
      \begin{enumerate}
        \item $w = \epsilon$: From the definition of bisimulation it is known
            that $\accepting{P_1} \iff \accepting{P_2}$.
          This is equivalent to
            $\accepting{\hat{\delta}_N(P_1, \epsilon)}
              \iff \accepting{\hat{\delta}_N(P_2, \epsilon))}$
            and concludes this point.
        \item $w = a \cdot v$:
          We make use of the induction hypothesis and bisimulation properties
          to get
            $\accepting{\hat{\delta}_N(\delta_N(P_1, a), v)}
              \iff \accepting{\hat{\delta}_N(\delta_N(P_2, a), v)}$
              (we move backwards through the string).
          This can be transformed to
            $\accepting{\hat{\delta}_N(P_1, a \cdot v)}
              \iff \accepting{\hat{\delta}_N(P_2, a \cdot v)}$,
            concluding also this case.
      \end{enumerate}
\end{enumerate}
\end{proof}

\subsection{Computing bisimulations}

Bisimulations on states of a deterministic automaton can be computed using the
  following algorithm.
In order to use it to compare two distinct automata, the automata have to be merged.
It is necessary to ensure that their alphabets are the same (easily satisfied by
  using the union of the two alphabets), that their state sets are disjunct
  (by renaming), and that each of the input automata has only one starting state.

\begin{definition}
  This is the naive algorithm for deciding equivalence of two states of an automaton.
  \begin{align*}
    \underline{\mathit{Naive(p, q)}}: &\quad Q \to Q \to \{\mathbf{true}, \mathbf{false}\} \\
    \text{(1) } & R := \emptyset, \mathit{todo} := \{(p, q)\} \\
    \text{(2) } & \text{while } \mathit{todo} \neq \emptyset \text{ do}\\
      & \text{(2.1)}\quad \mathit{todo} := \mathit{todo} \setminus \{(p', q')\}\\
      & \textbf{(2.2)}\quad \textbf{if } \mathbf{(p', q') \in} \textbf{ R} \textbf{ then goto (2)}\\
      & \text{(2.3)}\quad \text{if } p' \in F \nLeftrightarrow q' \in F \text{ then return } \mathbf{false}\\
      & \text{(2.4)}\quad \mathit{todo} := \mathit{todo} \cup \bigcup_{a \in \Sigma}{(\delta(p', a), \delta(q', a))}\\
      & \text{(2.5)}\quad R := R \cup \{(p', q')\} \\
    \text{(3) } & \text{return } \mathbf{true}\\
  \end{align*}
\end{definition}


The algorithm first computes a bisimulation.
If it succeeds then the languages are equal.
If it fails to do so, the languages are not equal.
Finding a bisimulation, and therefore proving the language equivalence, yields
  the worst-case runtime of the algorithm.

%The algorithm can be extended to also find a witness for the inequality of the
%input states. This can be done by tagging each pair in \textit{todo} with the
%symbol sequence which yielded it. This results in the following algorithm,
%which yields either \textbf{true} or a word $w$ for which
%$\neg (\hat{\delta}(p, w) \in F \iff \hat{\delta}(q, w) \in F)$.
%
%\begin{definition}
  \begin{align*}
    \underline{\mathit{NaiveWithWitness(p, q)}}: &\quad Q \to Q \to \{\mathbf{true}, \mathbf{\Sigma^\ast}\} \\
    \text{(1) } & R := \emptyset, \mathit{todo} := \{(p, q)_\epsilon\} \\
    \text{(2) } & \text{while } \mathit{todo} \neq \emptyset \text{ do}\\
      & \text{(2.1)}\quad \mathit{todo} := \mathit{todo} \setminus \{(p', q')_w\}\\
      & \text{(2.2)}\quad \text{if } (p', q') \in \text{ R} \text{ then goto (2)}\\
      & \text{(2.3)}\quad \text{if } p' \in F \nLeftrightarrow q' \in F \text{ then return } w\\
      & \text{(2.4)}\quad \mathit{todo} := \mathit{todo} \cup \bigcup_{a \in \Sigma}{(\delta(p', a), \delta(q', a))_{wa}}\\
      & \text{(2.5)}\quad R := R \cup \{(p', q')\} \\
    \text{(3) } & \text{return } \mathbf{true}\\
  \end{align*}
\end{definition}


\subsection{Bisimulations up-to}

Since in the worst case there are $\mathcal{O}(n^2)$ pairs of states of the
  input automaton, it is advantageous to think about how to reduce the number of
  pairs we have to compute.

\begin{definition}
A relation R is a bisimulation up-to $f$
($f : \mathcal{P}(Q \times Q) \to \mathcal{P}(Q \times Q$)) iff $R \rightarrowtail f(R)$.
\end{definition}

Sangiorgi introduced the notion of \emph{respectful} functions to describe
  function which yield bisimulations up-to.
In an earlier work of Pous he introduced \emph{compatibility} as a generalisation of
  Sangiorgi's respectfulness~\cite{pous2007complete}, but in the context of Bonchi and Pous's paper they
  are equivalent.
Nevertheless, Bonchi and Pous are using this name in their work.

\begin{definition}
  A function $f : \mathcal{P}(Q \times Q) \to \mathcal{P}(Q \times Q)$ is compatible if it is
  \begin{itemize}
    \item monotone ($\forall R, R' \subseteq Q \times Q: R \subseteq R' \Rightarrow f(R) \subseteq f(R')$), and
    \item preserves progression:
        $\forall R, R' \subseteq Q \times Q: R \rightarrowtail R' \Rightarrow f(R) \rightarrowtail f(R')$.
  \end{itemize}
\end{definition}

A relation $R$ is a \emph{bisimulation up-to} $f$ iff $f(R)$ is a bisimulation.
The function $f$ has to be compatible. Bonchi and Pous proved that
  any bisimulation up-to is contained in a bisimulation.
This is what their work is about:
  making it possible to more easily compute bisimulations.
This is because bisimulations up-to are often smaller than full bisimulations
  because $f$ is mototone.

A main contribution of Bonchi and Pous is also to show that many nontrivial
  and useful functions are compatible.
The most important of them are

\begin{itemize}
  \item the identity function
  \item pointwise union of a family of compatible functions: \\
          $\bigcup F(R) = \bigcup_{f \in F} f(R)$
  \item $f \circ g$ where $f$ and $g$ are compatible function
  \item the identity closure of $R \subseteq \mathcal{P}(Q \times Q): R \cup \{(q, q)\forWhich q \in Q\}$
  \item the symmetric closure of a relation: $s(R) = R \cup \{(q, p) \forWhich p \setAsRelation{R} q\}$
  \item squaring of a relation $R$: $P \setAsRelation{R} P' \land P' \setAsRelation{R} P'' \Rightarrow P \setAsRelation{R} P''$
  \item infinite iteration of a compatible function: $f^\omega(R) = \bigcup_{n = 1}^{\infty} f^n(R)$
  \item by the above four points, the equivalence closure
\end{itemize}

\begin{example}
  We show that the identity function is compatible.
  It is obvious that the identity function is monotone.
  To show progression it is sufficient to insert the identity function into the
    statement.
  We get
    $\forall R, R' \subseteq Q \times Q: R \rightarrowtail R'
      \Rightarrow R \rightarrowtail R'$,
  which is trivially true.
\end{example}


\subsection{Examples}

Hopcroft and Karp's algorithm (HK) for determining the equivalence of DFAs is
  equivalent to computing a bisimulation up to equivalence, that is, exploiting
  symmetry and transitivity.

Bonchi and Pous exploit the following property of the states of determinised NFAs:

\begin{corollary}
  Computing the language of a determinised NFA is a homomorphism:\\
  $\mathcal{L}(P_1) = \mathcal{L}(Q_1) \land \mathcal{L}(P_2) = \mathcal{L}(Q_2)
    \Rightarrow \mathcal{L}(P_1 \cup P_2) = \mathcal{L}(Q_1 \cup Q_2)$
\end{corollary}

After proving that this function is compatible a
  \emph{bisimulation up to congruence} can be computed for determinised NFAs.
The \emph{congruence closure} function combines the equational closure with
  the inference rule specified by the above corollary.

A general problem which occurs when we extend HK to determinised NFAs is the
size of the necessary union-find structure.
Since the states are the powerset of the underlying NFA, the data structure
will have to store too many states.
Using the congruence closure only aggravates the problem.

Bonchi and Pous instead define a notion of \emph{set rewriting}.
This allows replacing subsets of sets according to approproate rewrite rules.
Bonchi and Pous also proved confluence for the rewrite rules generated by
the congruence closure.
To exploit this for their data structure, they simply store the normal forms of
the pairs contained in them.
When membership of a pair is requested, the normal forms are computed and
looked up.
