\section{Experimental results}

\subsection{Experiment setup}

Bonchi and Pous conducted two experiments to compare Hopcroft and Karp's original
algorithm (HK), their improved algorithm (HKC and HKC'),
and the antichain algorithms (AC and AC').
The first experiment measures the performance of checking language equivalence,
while the second experiment is about language inclusion.

HK, HKC, and HKC' were implemented by Bonchi and Pous in OCaml and was published
online along with a web interface which visualizes the operation of all
algorithms~\cite{hkntonline}.

The antichain algorithms were represented by the VATA library,
implemented in C++~\cite{lengal2012vata}.
This was the most efficient implementation of these algorithms known to
Bonchi and Pous at the time of the experiments.
Though designed for tree automata, they can also be used to compare NFAs
by representing them as unary tree automata.

Language equivalence was checked for one thousand random automata with varying
number of states.
Bonchi and Pous took care to test the worst-case behaviour of the algorithms by
not generating automata with accepting states.
This ensures that there is no possible pair of accepting and non-accepting
states, and thus any algorithm will be forced to analyze any relevant pair.
Also, the automata are generated such that the expected number of
state transitions per state and input alphabet symbol is $1.25$, which
statistically maximizes the size of the corresponding minimized DFA, according
to Tabakov and Vardi~\cite{Tabakov2005}.

Language inclusion examples are taken from model checking of programs.
They were initially used by Abdullah et al.\ in~\cite{abdulla2010simulation}
to evaluate their improvements to the antichain algorithm.

\subsection{Results}



\subsection{Analysis}

For the equivalence case it is quite clear that HKC outperforms both HK and AC
by orders of magnitude.
For the case of $n = 1000$ AC seems to be able to keep up with HKC, but
there are also cases where AC runs out of memory, while HKC does not.
The reason why is the large number of states that AC has to compare, and, thus,
represent in memory.

Since the similarity-enhanced algorithms AC' and HKC' require preprocessing time
to compute the similarity relation, the time needed for this step has to be
taken into account when comparing runtimes.
The language inclusion experiment shows that in the inclusion case AC and HKC
are outperformed by the similiarity-enhanced versions.
In the counterexample cases the similarity-enhanced algorithms are slower,
though there are cases when it pays off to use them.
It is hard to interpret the results of the equivalence experiments because
Bonchi and Pous did not include the time required to calculate the equality
relation in the results.
All in all it can be said that the similarity-enhanced versions of the
algorithms perform better, but they depend on a precomputed similarity relation.

Since VATA is written for tree automata, word automata have to be represented
as working on terms composed of unary function symbols.
This could have an impact on the processing time of the algorithms.
To get a fairer comparison of the algorithms it would be useful to implement
them in the same language and make them use the same libraries and data
representations and then repeat the experiments.
